\documentclass{article}
\usepackage[utf8]{inputenc}
\usepackage{geometry}
\geometry{
    a4paper,
    total={170mm,257mm},
    left=20mm,
    top=20mm,
}
\usepackage{graphicx}
\usepackage{titling}
\usepackage{fancyhdr}

\fancypagestyle{plain}{%  the preset of fancyhdr 
    \fancyhf{} % clear all header and footer fields
    % \fancyfoot[R]{\includegraphics[width=2cm]{KULEUVEN_GENT_RGB_LOGO.png}}
    \fancyfoot[L]{Jar explorR}
    \fancyhead[L]{Programozás Alapjai 3. Nagy Házi feladat}
    \fancyhead[R]{Ferencz Péter}
}
% \makeatletter

\usepackage{lipsum}
\usepackage{cmbright}

\begin{document}

\begin{titlepage}
    \centering
    \vspace{1cm}
    {\Large \textsc{Programozás Alapjai 3\\(BMEVIIIAB00, 2025/26/1)\\  Nagy házi feladat}\par}
    \vspace{1.5cm}
    {\huge\bfseries Jar explorR:\\Java Archive fájl elemző és UML diagram generátor\par}
    \vspace{2cm}
    {\Large\itshape készítette: \par Ferencz Péter\par}
    {\large(RFG7SN)}
    \vfill
    \includegraphics[totalheight=3cm]{res/VIK.png}\par
    \vspace{1cm}
    \includegraphics[totalheight=3cm]{res/BMEKicsi.png}\par
    Budapesti Műszaki és Gazdaságtudományi Egyetem\par
    Villamosmérnöki és Informatikai Kar\par
    Mérnökinformatikus Bsc\par
    {\large 2025 Október-November\par}
\end{titlepage}


\section*{A programról}
A program célja egy Java Archive fájl vizuális ismertetése, az osztályok függvényeinek és metódusainak kigyűjtése, rendszerezése.
UML diagramm autómatikus elkészítése.

\section*{Bemenet}
A program fő célja a JAR fájlok kezelése, ezért bemenetként egy ilyen fájlt vár. Ezt megkaphatja a \textit{\textendash input [file.jar]} kapcsoló
paramétereként, melyet a grafikus felület megjelenése után meg is nyit, vagy a \textit{File \textgreater Open} menüpont alatt választhatja ki
az operációs rendszer fájlböngészőjéből a megnyitni kívánt fájlt.

\section*{Grafikus felület}
A program 3 fő grafikai elemből áll: Fa nézet, Füles nézet, Eszköztár.\\
A fa nézet célja a jar fájl fizikai felépítését szemléltetni, ahol minden egyes csomag (package) a fa egy új
szintjén helyezkedik el. A fa levelei az osztály (.class) fájlok. Emellett a program megjeleníti az egyéb állományokat is,
ezek megjelenítésére viszont képtelen (ezalól kivételt képez a META-INF/MANIFEST.MF).\\
A füles nézet (tabbed layout) lehetővé teszi, hogy több osztályfájlt megnyissunk, ezek között könnyen váltogassunk.
A fa nézetben az osztályfájlra kattintva megnyílik az adott osztályról információkat tartalmazó panel. A panel tartalmazza
a fájlban fellelhető összes metódust, változót, azok hívási aláírását, és egyéb osztályadatokat.\\
Az eszköztár teszi lehetővé a program egyedi funkióinak megvalósítását. Menüpontjai:\\
File-Open: Egy grafikus fájlválasztót nyit meg, amin a felhasználó ki tudja választani az elemzendő jar fájlt.\\
File-Exit: A program kilép\\
UML-Show Diagram: megnyitja a füles nézetben az uml diagram szerkesztőt\\
UML-Save: egy képet tud elmenteni a felhasználó az elkészített diagramjáról

\section*{UML diagram}
Az UML diagram panel mozgatható jobb egér gombbal és kicsinyíthető/nagyítható az egérgörgővel.
Az egyes elemek (osztályok / interfészek) emellett szintén mozgathatók a bal egér gombbal.
Az uml diagram az alábbi jelöléseket mutatja:\\
Osztályok, interfészek és struktúrák külön jelölése\\
Öröklés és implementálás jelölése\\
Osztályok változóinak és metódusainak megjelenítése.

\end{document}